\documentclass[11pt]{article}
\pagestyle{empty}
\usepackage[utf8]{inputenc}
\usepackage{a4wide}
\usepackage{amsmath}
\usepackage{amssymb}
\usepackage{amsthm}
\usepackage{german}
\usepackage{graphicx}
%\usepackage{units}
\usepackage[locale=DE]{siunitx}
\usepackage{setspace}
\usepackage{threeparttable}
%\usepackage{url} 
\usepackage[hyphens]{url}
\usepackage{pdfpages}
\usepackage{ulem}
\usepackage{multirow}
\usepackage{hyperref}
\usepackage{polynom}
\usepackage{enumitem}
%\usepackage{ipe}
\usepackage{scrlayer-scrpage}
\usepackage{qtree}
\usepackage{cancel}

\usepackage{tikz}
\pagestyle{scrheadings}
\clearpairofpagestyles
\parindent0mm
\sloppy

% Typesetting code, setup for C by default
\usepackage{xcolor}
\usepackage{listings}
\lstdefinestyle{default}{%
  numbers=left,
  stepnumber=1,
  numberstyle=\tiny,
  basicstyle=\ttfamily,
  backgroundcolor=\color{gray!8},
  commentstyle=\color{green!60!blue}\itshape,
  keywordstyle=\color{blue},
  stringstyle=\color{blue!30!red},
  tabsize=4,
  keepspaces=true,
}
\lstset{style=default, language=Assembler}

% Basic data
\newcommand{\VORLESUNG}{TI2: Rechnerarchitektur}
\newcommand{\STUDENTS}{Bruno Stendal, Martin Baer, Lukas Gewinner und Christian Schäfer}
\newcommand{\STAFF}{Bernadette Keßler}
\newcommand{\ASSIGNMENT}{2}
\newcommand{\DELIVER}{Freitag, den 18.11.2022, 10:15 Uhr}
\setcounter{secnumdepth}{0}


% Arbitrary packages and settings

\newcommand{\N}{\mathbb{N}}
\newcommand{\cat}{++}
\newcommand{\lam}{\lambda}
\newcommand{\floor}[1]{\lfloor{#1}\rfloor}
\newcommand{\ceil}[1]{\lceil{#1}\rceil}
\newcommand{\half}[1]{\frac{#1}{2}}
\newcommand{\punkte}[1]{{\small{ }(#1 Punkte)}}

\newcommand{\aufgabe}[1]{\item{\bf #1}}

\begin{document}
% Document title
\ofoot{\pagemark}
\begin{center}
    Abgabe von \STUDENTS{}\\
 \ASSIGNMENT{}. Aufgabenblatt  zum Kurs 
\vspace*{0.2cm}

{\Large \VORLESUNG{}}

{\small von \STAFF{} \\ bis \DELIVER{}.}
\vspace*{0.5cm}\\
\end{center}
\section{Zahlenbasis}

\aufgabe{Lösen die folgenden Umrechnungen mit erkennbarem Lösungsweg und kommentieren Sie ggf. Ihre Annahmen oder Probleme.}
\begin{center}
        \includegraphics[width = 16cm,angle = 270]{TI2_2/A1_1.jpeg}
        \includegraphics[width = 20cm,angle = 270]{TI2_2/A1_2.jpeg}
    \end{center}
\newpage
\section{Pseudocode}
\aufgabe{}Übersetzung des C Codes in Pseudo Assembler
\lstinputlisting[language=C]{pseudocode_1.c}
\\
\lstinputlisting[language=C]{pseudocode_2.c}
\newpage
\section{Collatz Conjecture}
\aufgabe{}Programm Code wie folgt: 
\lstinputlisting[language=Assembler]{collatz.asm}
\end{document} 