\documentclass[11pt]{article}
\pagestyle{empty}
\usepackage[utf8]{inputenc}
\usepackage{a4wide}
\usepackage{amsmath}
\usepackage{amssymb}
\usepackage{amsthm}
\usepackage{german}
\usepackage{graphicx}
%\usepackage{units}
\usepackage[locale=DE]{siunitx}
\usepackage{setspace}
\usepackage{threeparttable}
%\usepackage{url} 
\usepackage[hyphens]{url}
\usepackage{pdfpages}
\usepackage{ulem}
\usepackage{multirow}
\usepackage{hyperref}
\usepackage{polynom}
\usepackage{enumitem}
%\usepackage{ipe}
\usepackage{scrlayer-scrpage}
\usepackage{qtree}
\usepackage{cancel}

\usepackage{tikz}
\pagestyle{scrheadings}
\clearpairofpagestyles
\parindent0mm
\sloppy

% Typesetting code, setup for C by default
\usepackage{xcolor}
\usepackage{listings}
\lstdefinestyle{default}{%
  numbers=left,
  stepnumber=1,
  numberstyle=\tiny,
  basicstyle=\ttfamily,
  backgroundcolor=\color{gray!8},
  commentstyle=\color{green!60!blue}\itshape,
  keywordstyle=\color{blue},
  stringstyle=\color{blue!30!red},
  tabsize=4,
  keepspaces=true,
}
\lstset{style=default, language=C}

% Basic data
\newcommand{\VORLESUNG}{TI2: Rechnerarchitektur}
\newcommand{\STUDENTS}{Bruno Stendal, Martin Baer, Lukas Gewinner und Christian Schäfer}
\newcommand{\STAFF}{Bernadette Keßler}
\newcommand{\ASSIGNMENT}{6}
\newcommand{\DELIVER}{Freitag, den 16.12.2022, 10:15 Uhr}
\setcounter{secnumdepth}{0}

% Arbitrary packages and settings

\newcommand{\N}{\mathbb{N}}
\newcommand{\cat}{++}
\newcommand{\lam}{\lambda}
\newcommand{\floor}[1]{\lfloor{#1}\rfloor}
\newcommand{\ceil}[1]{\lceil{#1}\rceil}
\newcommand{\half}[1]{\frac{#1}{2}}
\newcommand{\punkte}[1]{{\small{ }(#1 Punkte)}}

\newcommand{\aufgabe}[1]{\item{\bf #1}}

\begin{document}
% Document title
\ofoot{\pagemark}
\begin{center}
    Abgabe von \STUDENTS{}\\
 \ASSIGNMENT{}. Aufgabenblatt  zum Kurs 
\vspace*{0.2cm}

{\Large \VORLESUNG{}}

{\small von \STAFF{} \\ bis \DELIVER{}.}
\vspace*{0.5cm}\\
\end{center}

\section{Fließbandverarbeitung}
\aufgabe{Sorgen Sie dafür dass die folgende Befehlsfolge aus Pseudoinstructions in den jeweiligen Pipelines konfliktfrei ausgeführt wird.}\\
Programmcodeunterteilung:
\begin{enumerate}
    \item add r0, r1, r2
    \item add r1, r5, r0
    \item mov [rsp+8], r5
    \item or r0, r5, r4
    \item mov r3, [rsp+24]
    \item and r1, r0, r3
    \item add r0, r1, r3
    \item add r0, r0, 0x341D
    \item add r0, r0, 0x52F6
\end{enumerate}
\aufgabe{Einfache Pipeline\\ Gehen Sie hier von einer einfachen 5-stufigen Pipeline aus: Befehl holen (IF), Befehl dekodieren (ID), Operanden holen (OF), Ausführung (EX), Rückspeichern (WB). Weiterhin liegt eine reine Load/Store-Architektur ohne architekturelle Beschleunigungsmaßnahmen (z.B. Forwarding, Reordering etc.) oder Hardware zur Erkennung von Hemmnissen vor. Operanden können erst dann aus Registern geholt werden, nachdem sie zurück gespeichert wurden.}\\
\begin{center}
\begin{tabular}{c|c|c|c|c|c}
    Step & IF & ID & OF & EX & WB \\
    \hline
    1. & 1 & NOP & NOP & NOP & NOP \\
    2. & NOP & 1 & NOP & NOP & NOP \\
    3. & NOP & NOP & 1 & NOP & NOP \\
    4. & 2 & NOP & NOP & 1 & NOP \\
    5. & NOP & 2 & NOP & NOP & 1 \\
    6. & NOP & NOP & 2 & NOP & NOP \\
    7. & 3 & NOP & NOP & 2 & NOP \\
    8. & NOP & 3 & NOP & NOP & 2\\
    9. & NOP & NOP & 3 & NOP & NOP \\
    10. & 4 & NOP & NOP & 3 & NOP \\
    11. & NOP & 4 & NOP & NOP & 3 \\
    12. & NOP & NOP & 4 & NOP & NOP \\
    13. & 5 & NOP & NOP & 4 & NOP \\
    14. & NOP & 5 & NOP & NOP & 4 \\
    15. & NOP & NOP & 5 & NOP & NOP \\
    16. & 6 & NOP & NOP & 5 & NOP \\
    17. & NOP & 6 & NOP & NOP & 5 \\
    18. & NOP & NOP & 6 & NOP & NOP \\
    19. & 7 & NOP & NOP & 6 & NOP \\
    20. & NOP & 7 & NOP & NOP & 6 \\
    21. & NOP & NOP & 7 & NOP & NOP \\
    22. & 8 & NOP & NOP & 7 & NOP \\
    23. & NOP & 8 & NOP & NOP & 7 \\
    24. & NOP & NOP & 8 & NOP & NOP \\
    25. & 9 & NOP & NOP & 8 & NOP \\
    26. & NOP & 9 & NOP & NOP & 8 \\
    27. & NOP & NOP & 9 & NOP & NOP \\
    28. & NOP & NOP & NOP & 9 & NOP \\
    29. & NOP & NOP & NOP & NOP & 9 \\
    30. &  &  & Ende. &  & 
\end{tabular}
\end{center}
\aufgabe{Verbesserte Pipeline \\ Gehen Sie jetzt davon aus, dass die Pipeline aus dem vorherigen Aufgabenteil die in der Vorlesung kennengelernten Forwarding-Varianten/Shortcuts verwendet.}
\begin{center}
\begin{tabular}{c|c|c|c|c|c}
    Step & IF & ID & OF & EX & WB \\
    \hline
    1. & 1 & NOP & NOP & NOP & NOP \\
    2. & 2 & 1 & NOP & NOP & NOP \\
    3. & 3 & 2 & 1 & NOP & NOP \\
    4. & 3 & 2 & NOP & 1 & NOP \\
    5. & NOP & 3 & 2 & NOP & 1 \\
    6. & 4 & NOP & 3 & 2 & NOP \\
    7. & 5 & 4 & NOP & 3 & 2 \\
    8. & 6 & 5 & 4 & NOP & 3 \\
    9. & NOP & 6 & 5 & 4 & NOP \\
    10. & 7 & NOP & 6 & 5 & 4 \\
    11. & NOP & 7 & NOP & 6 & 5 \\
    12. & 8 & NOP & 7 & NOP & 6 \\
    13. & NOP & 8 & NOP & 7 & NOP \\
    14. & 9 & NOP & 8 & NOP & 7 \\
    15. & NOP & 9 & NOP & 8 & NOP \\
    16. & NOP & NOP & 9 & NOP & 8 \\
    17. & NOP & NOP & NOP & 9 & NOP \\
    18. & NOP & NOP & NOP & NOP & 9 \\
    19. &  &  & Ende. &  & 
\end{tabular}
\end{center}
\section{Arithmetik}
\aufgabe{Integer-Arithmetik \\ Implementieren Sie eine Funktion, die den ganzzahliegen Anteil der Formel berechnet.\\ Fließkomma-Arithmetik \\ Implementieren Sie die obige Formel für Fließkommazahlen.}

\lstinputlisting[language=Assembler]{formula.asm}
\newline
Teil 1
\newline
\begin{itemize}
    \item g und h liegen in [rsp+8] und [rsp+16]
    \item Die division von negativen Zahlen ist nicht möglich daher müssen wir das Vorzeichen vor und nach der Divsion umkehren
    \item $2^{31}$, nicht 32 da wir ein signed Integer haben, hat einen maximalen Zahlenwert von 2.147.483.648 was wenn ein Paramter über 2 Milliarden ist überschritten werden würde und einen Overflow produzieren würde.
\end{itemize}
\newline
Teil 2
\newline
\begin{itemize}
    \item Die xmm Register von xmm0 bis xmm15
    \item addsd, divsd, mulsd und subsd, wobei das s am ende für scalar steht und das d für double precision also 64bit steht.
    \item Alle xmm Register gelten als Volatile und werden bei einem Funktionsaufrum genullt.
\end{itemize}
\end{document} 